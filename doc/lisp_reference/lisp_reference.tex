\documentclass[letterpaper,11pt,twocolumn]{article}
\usepackage[top=2cm, bottom=2cm, left=2cm, right=2cm]{geometry}\usepackage[T1]{fontenc}
\usepackage[utf8]{inputenc}
\usepackage{lmodern}
\usepackage{amsmath}
\usepackage{amsfonts}
\usepackage{amssymb}
\usepackage{amsthm}
\usepackage{graphicx}
\usepackage{color}
\usepackage{xcolor}
\usepackage{url}
\usepackage{textcomp}
\usepackage{listings}
\usepackage{glossaries}
\usepackage{parskip}
\usepackage{imakeidx}
\usepackage[normalem]{ulem}

\title{Bluejay Lisp Reference}
\author{swissChili}
\date{Development Version: \today}

\newcommand{\startexplanation}{$\triangleright$\hskip1.4ex}
\newenvironment{defin}[1]
               {#1
               \begin{quote}\startexplanation}
               {\end{quote}}

\newcommand{\plus}{\texttt{+}}
\newcommand{\pluses}[1]{\plus{}#1\plus{}}
\newcommand{\earmuff}{\texttt{*}}
\newcommand{\earmuffs}[1]{\earmuff{}#1\earmuff{}}
\newcommand{\func}[1]{\text{$_f$\textbf{#1}}}
\newcommand{\mac}[1]{\text{$_m$\textbf{#1}}}
\newcommand{\reader}[1]{\text{$_r$\textbf{#1}}}
\newcommand{\const}[1]{\text{$_c$\textbf{#1}}}
\newcommand{\var}[1]{\text{$_v$\textbf{#1}}}
\newcommand{\param}[1]{\textit{#1}}
\newcommand{\ret}[1]{\uline{#1}}
\newcommand{\type}[1]{\text{$_t$\textbf{#1}}}
\newcommand{\more}{ \ldots}
\newcommand{\T}{\texttt{t}}
\newcommand{\nil}{\texttt{nil}}
\newcommand{\default}[1]{\text{\textsubscript{
      \setlength{\fboxsep}{1pt}\setlength{\fboxrule}{0.2bp}%
      \fbox{#1}}}}
\newcommand{\opt}[2]{\text{$[$}\param{#1}\default{#2}\text{$]$}}
\newcommand{\mut}[1]{\text{$\widetilde{#1}$}}
\newcommand{\mighteval}[1]{\text{$\widehat{#1}$}}
\newcommand{\super}[1]{\text{$ ^{#1} $}}

\newcommand{\optlist}[1]{\text{\(
    \left\{
      \begin{array}{l}
        #1
      \end{array}
    \right\}
    \)}}

\makeindex
\makeglossaries

\newglossaryentry{closure}{name={closure},description={A
    \type{function-object} that captures certain variables from its
    defining context}}

\newglossaryentry{lexical-scope}{name={lexical scope},description={A
    method of scoping where the values in scope at a given time are
    determined by the static content of the source program, not by the
    state of the callstack or other execution details. This allows for
    example closures to \gls{closure}s to capture some of the state of
    their defining context}}

\newglossaryentry{truthy}{name={truthy},description={A value that is
    not \nil}}

\begin{document}

\maketitle
\tableofcontents

\section{Introduction}

This document provides a brief reference to the Bluejay Lisp language
and standard library. It documents every currently included function,
macro, reader macro, and special form.

\subsection{Running Bluejay Lisp}

Assuming you have compiled Bluejay Lisp, you should have a
\texttt{lisp} executable in your \texttt{src/lisp} directory. Invoking
\texttt{lisp} without arguments will drop you into an interactive
read-evaluate-print loop, or REPL.\@ Here you can type Lisp code one
line at a time---the result will be immediately displayed. If you wish
to run a specific Lisp file, you may pass its path as the first
argument to \texttt{lisp}, e.g. \texttt{lisp hello-world.lisp}.

Note that in order to load the standard library, the compiler must
know where Bluejay Lisp libraries are stored. When loading a library,
the compiler checks the \texttt{LISP\_LIBRARY\_PATH} environment
variable. If it is blank, it assumes the library path is
\texttt{/lib/lisp}. To use the standard library included with Bluejay,
set the environment variable to the path of the included
\texttt{lib/lisp} directory.

\subsection{Typography}

The following text styles and symbols are used within this document to
indicate particular types, values, or meanings:

\begin{tabular}[t]{p{0.2\linewidth} p{0.64\linewidth}}
  \func{cons} & A function \param{cons}. \\
  \const{\plus{}foo\plus} & A constant \param{\plus{}foo\plus}. \\
  \var{\earmuff{}bar\earmuff} & A global variable \param{\earmuff{}bar\earmuff}. \\
  \reader{baz} & A reader macro \param{baz}. \\
  \mac{quux} & A macro \param{quux}. \\
  \param{parameter} & A function argument \param{parameter} \\
  \opt{var}{123} & An optional function argument \param{var} with the default value \param{123}. \\
  \param{args}\more & \param{args} represents the rest of the items in the list. \\
  \mut{\param{mut}} & A function argument \param{mut} that might be mutated. \\
  \mighteval{\param{maybe}} & \param{maybe} may or may not be evaluated. \\
  \ret{value} & Indicates that a form will evaluate to \param{value}. \\
  \type{integer} & The type \param{integer}. \\
  \optlist{\text{a}\\\text{b}} & One of \param{a} or \param{b}.
\end{tabular}

\section{Primitives}

\subsection{Type Predicates}

\begin{defin}{
    (\optlist{
      \func{nilp} \text{ or } \func{not} \\
      \func{closurep} \text{ or } \func{functionp} \\
      \func{integerp} \\
      \func{consp} \\
      \func{symbolp}
    } \param{value})\index{nilp}\index{not}
  }
  \ret{\T} if \param{value} is of the specified type, \ret{\nil}
  otherwise.
\end{defin}

\begin{defin}{(\func{listp} \param{value})}
  \ret{\T} if \param{value} is a well-formed \type{cons} list,
  i.e.\ if (\func{consp} \param{value}) and the (\func{cdr}
  \param{value}) is a well-formed \type{cons} list or \nil.
\end{defin}

\subsection{Definitions and Variables}

\begin{defin}{
    (\mac{defun} \param{name} (\param{args}\more) \param{body}\more)\index{defun} \\
    (\mac{defmacro} \param{name} (\param{args}\more) \param{body}\more)\index{defmacro}
  }
  Define a function or macro respectively, taking \param{args}
  arguments and evaluating \param{body} in turn, finally evaluating to
  \ret{the final entry in \param{body}} or \ret{\nil} if \param{body}
  is empty.
\end{defin}

\begin{defin}{
    (\mac{let1} (\param{variable} \param{form}) \param{body}\more)\index{let1} \\
    (\mac{let} ((\param{variable} \param{form})\more) \param{body}\more)\index{let}
  }
  First evaluate \param{form}, binding it to \param{variable}. Then
  evaluate \param{body}, finally evaluating to \ret{the final entry in
    \param{body}} or \ret{\nil} if \param{body} is
  empty. \param{variable} is no longer in scope after this form ends.
  \mac{let} is similar to \mac{let*} in other lisps, later variables
  can reference previous ones.
\end{defin}

\begin{defin}{
    (\mac{flet1} (\param{name} (\param{args}\more) \param{body}\more)) \\
    (\mac{flet} ((\param{name} (\param{args}\more)\more) \param{body}\more))
    \index{flet}\index{flet1}
  }
  Like \mac{let} and \mac{let1} but creates a lambda. Unlike other
  lisps the defined function remains in the variable namespace as a
  \type{function-object}.
\end{defin}

\begin{defin}{
    \reader{\textquotesingle}\param{value} \\ (\mac{quote}
    \param{value})\index{quote} } \ret{Return \param{value}} as an
  expression, without evaluating it.
\end{defin}

\begin{defin}{
    \reader{\`}\param{value} \\
    (\mac{backquote} \param{value})\index{backquote}\index{\`}
  }
  Return \ret{\param{value}} unevaluated, except for any \mac{unquote}
  and \mac{unquote-splice} forms (and their reader-macro equivalents).
\end{defin}

\begin{defin}{
    \reader{,}\param{value} \\
    (\mac{unquote} \param{value}) \\
    \reader{,\texttt{@}}\param{value} \\
    (\mac{unquote-splice} \param{value})
    \index{unquote}\index{unquote-splice}
    \index{,}\index{,@}
  }
  In the context of a \mac{backquote}, evaluate \param{value} instead of
  using it as-is. \mac{unquote-splice} and \reader{,\texttt{@}} take a
  list as their argument, and ``splice'' each element in to
  the \mac{backquote} list so that each element of that argument corresponds
  to an element of the \mac{backquote} list.
  The other two macros insert their argument as a single element in the list.
\end{defin}

\subsection{Control Flow}

\begin{defin}{
    (\mac{if} \param{predicate} \param{then} \opt{otherwise}{nil})\index{if} \\
    (\mac{when} \param{predicate} \param{then\more})\index{when} \\
    (\mac{unless} \param{predicate} \param{otherwise\more})\index{unless}
  }
  First evaluate \param{predicate}. If it is \gls{truthy} evaluate and
  return \ret{\param{then}}, otherwise \ret{\param{otherwise}}. If
  either is not provided return \ret{\nil}.
\end{defin}

\begin{defin}{
    (\mac{progn} \opt{forms\more}{nil})\index{progn}
  }
  Evaluate \param{forms} from first to last, finally returning
  \ret{the last form}.
\end{defin}

\begin{defin}{
    (\mac{and} \param{first} \param{\mighteval{rest}}\more)\index{and} \\
    (\mac{or} \param{first} \param{\mighteval{rest}}\more)\index{or}
  }
  Short circuiting $\land$ and $\lor$, respectively. Return the first
  value that is \nil{} or truthy, respectively, or the last value if
  all are truthy/\nil{}.
\end{defin}

\section{Numbers}

\subsection{Integers}

\begin{defin}{
    (\optlist{
      \func{$+$} \\
      \func{$-$} \\
      \func{$*$} \\
      \func{$/$}
    } \param{a b})
  }
  The \ret{sum, difference, product, or quotient} of \param{a} and
  \param{b} as an \type{integer}.
\end{defin}

\begin{defin}{
    (\func{=} \param{a b})
  }
  \ret{\T} if \param{a} and \param{b} hold the same \type{integer}
  value, \ret{\nil} otherwise.
\end{defin}

\begin{defin}{(\optlist{\func{$<$} \\ \func{$>$}} \param{a} \param{b})}
  \ret{\T} if \param{a} is less than or greater than \param{b},
  respectively; \ret{\nil} otherwise.
\end{defin}

\section{Function Manipulation}

\begin{defin}{
    (\func{funcall} \param{args}\more)\index{funcall} \\
    (\func{apply} \param{function} \param{args})\index{apply}
  }
  Call the \type{closure} or \type{function-object} \param{function}
  with \param{args} and evaluate to \ret{its result}. An error occurs
  if \param{args} are not acceptable.
\end{defin}

\begin{defin}{
    (\func{recurse} \param{args}\more)\index{recurse} } In a lambda
  definition, call the current lambda with
  \param{args}. \func{recurse} can also be used as an argument to
  \func{function} and \reader{\#\textquotesingle} to refer to the
  current lambda.
\end{defin}

\begin{defin}{
    (\mac{function} \param{function-name})\index{function} \\
    \reader{\#\textquotesingle}\param{function-name}\index{\#\textquotesingle}
  }
  Create a \ret{\type{function-object} from an existing function or
    macro}. \param{function} must be a symbol literal at compile time.
\end{defin}

\begin{defin}{
    (\func{lambda} (\param{args}\more) \param{body}\more)\index{lambda}
  }
  Create a \ret{lexically-scoped \type{\gls{closure}}} taking
  \param{args} and evaluating to \param{body}.
\end{defin}

\begin{defin}{
    (\func{eval} \param{form})\index{eval}
  }
  Evaluate and return \ret{\param{form}} in the current global
  environment. The evaluated form does not use \gls{lexical-scope}.
\end{defin}

\section{Lists}

\subsection{Creating Lists}

\begin{defin}{
    (\func{cons} \param{a} \param{b})\index{cons}
  }
  Join \param{a} and \param{b} into a \ret{\type{cons} pair}.
\end{defin}

\begin{defin}{
    (\func{list} \param{values}\more)\index{list}
  }
  Construct a \ret{\type{cons}-list of \param{values}}.
\end{defin}

\begin{defin}{(\func{append} \param{first} \param{rest})}
  If \param{first} is a list, return \ret{a list with all the elements
    of \param{first} followed by the elements of
    \param{rest}}. Otherwise, return \ret{(\func{cons} first rest)}.
\end{defin}

\subsection{Deconstructing Lists}

\begin{defin}{
    (\func{length} \param{list})\index{list}
  }
  Return the \ret{length of \param{list}} if it is a \type{cons}-list,
  \nil{} otherwise.
\end{defin}

\begin{defin}{
    (\func{car} \param{pair})\index{car} \\
    (\func{cdr} \param{pair})\index{cdr}
  }
  Return the \ret{first or second item} of \type{cons} \param{pair},
  respectively.
\end{defin}

\begin{defin}{
    (\optlist{
      \func{caar}\\
      \func{cadr}\\
      \func{caddr}\\
      \func{cadar}\\
      \func{caddar}
    } \param{val})
  }
  Behave like a combination of \func{car} and \func{cdr} would.
\end{defin}

\begin{defin}{
    (\func{elt} \param{list} \param{n})\index{elt}
  }
  Return the \ret{\param{n}\super{th} element of \param{list}},
  starting from 0, or \ret{\nil} if \param{n} $ \ge $ (\func{length}
  \param{list}) or \param{list} is not a \type{cons}-list or \param{n}
  is not an \type{integer}.
\end{defin}

\subsection{Operating on Lists}

\begin{defin}{
    (\func{mapcar} \param{fun} \param{list})\index{mapcar}
  }
  Apply \param{fun} to each element of \param{list}, returning a
  \ret{new \type{cons}-list} containing the results of the respective
  applications of \param{fun}.
\end{defin}

\begin{defin}{
    (\optlist{\func{remove-if}\\\func{remove-if-not}}
    \param{predicate} \param{list})\index{remove-if}\index{remove-if-not}
  }
  Return a \ret{new \type{cons}-list} of all the items of \param{list}
  that either do not or do satisfy \param{predicate}, respectively.
\end{defin}

\begin{defin}{
    (\func{reduce} \param{fun} \param{list} \opt{initial-value}{\nil})
  }
  Apply \param{fun} to two arguments at a time, starting with
  \param{initial-value} and (\func{car} \param{list}) and continuing
  with the result of the previous invocation and the successive
  element of \param{list}. Return \ret{the result of the final
    invocation}, or \ret{\param{initial-value}} if \param{list} is
  empty.
\end{defin}

\section{Input \& Output}

\begin{defin}{
    (\func{print} \param{value})\index{print}
  }
  Print \param{value} to standard output. Return \ret{\nil}.
\end{defin}

\begin{defin}{
    (\func{read})\index{read}
  }
  Read and return an \ret{S-expression} from standard input
\end{defin}

\subsection{Loading Programs}

\begin{defin}{
    \const{\pluses{current-file}}\index{\pluses{current-file}} }
  \ret{A string} containing the path of the file currently being
  compiled, or \ret{\nil{}} if not compiling a file (e.g.\ when
  compiling a string).
\end{defin}

\begin{defin}{
    (\func{load} \param{lisp-file})\index{load}
  }
  Load and evaluate \type{string} \param{lisp-file} as a local path
  relative to the current file, or the current working directory if
  not compiling a file. Return \ret{\nil}.
\end{defin}

\section{Inspecting \& Modifying the Environment}

\subsection{Managing Memory}

\begin{defin}{
    (\func{gc})\index{gc}
  }
  Run the garbage collector.
\end{defin}

\begin{defin}{
    (\func{gc-stats})\index{gc-stats}
  }
  Return some
  statistics about the garbage collector, in the form of a list:
  \ret{(\param{total-allocs} \param{gc-runs})}, where
  \param{total-allocs} is the current number of active allocations,
  and \param{gc-runs} is the total number of times the garbage
  collector has been run.
\end{defin}

\subsection{Introspection}

\begin{defin}{ \const{\pluses{current-env}} }
  An opaque object representing the current environment.
\end{defin}

\begin{defin}{ (\func{env-functions} \param{current-env}) }
  Returns a list of symbols, each of which is the name of a function
  defined in the environment specified by \param{current-env}.
\end{defin}

\printindex

\printglossaries

\end{document}
